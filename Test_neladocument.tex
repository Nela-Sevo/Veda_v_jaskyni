% Options for packages loaded elsewhere
% Options for packages loaded elsewhere
\PassOptionsToPackage{unicode}{hyperref}
\PassOptionsToPackage{hyphens}{url}
\PassOptionsToPackage{dvipsnames,svgnames,x11names}{xcolor}
%
\documentclass[
  slovak,
  letterpaper,
  DIV=11,
  numbers=noendperiod]{scrartcl}
\usepackage{xcolor}
\usepackage{amsmath,amssymb}
\setcounter{secnumdepth}{-\maxdimen} % remove section numbering
\usepackage{iftex}
\ifPDFTeX
  \usepackage[T1]{fontenc}
  \usepackage[utf8]{inputenc}
  \usepackage{textcomp} % provide euro and other symbols
\else % if luatex or xetex
  \usepackage{unicode-math} % this also loads fontspec
  \defaultfontfeatures{Scale=MatchLowercase}
  \defaultfontfeatures[\rmfamily]{Ligatures=TeX,Scale=1}
\fi
\usepackage{lmodern}
\ifPDFTeX\else
  % xetex/luatex font selection
\fi
% Use upquote if available, for straight quotes in verbatim environments
\IfFileExists{upquote.sty}{\usepackage{upquote}}{}
\IfFileExists{microtype.sty}{% use microtype if available
  \usepackage[]{microtype}
  \UseMicrotypeSet[protrusion]{basicmath} % disable protrusion for tt fonts
}{}
\makeatletter
\@ifundefined{KOMAClassName}{% if non-KOMA class
  \IfFileExists{parskip.sty}{%
    \usepackage{parskip}
  }{% else
    \setlength{\parindent}{0pt}
    \setlength{\parskip}{6pt plus 2pt minus 1pt}}
}{% if KOMA class
  \KOMAoptions{parskip=half}}
\makeatother
% Make \paragraph and \subparagraph free-standing
\makeatletter
\ifx\paragraph\undefined\else
  \let\oldparagraph\paragraph
  \renewcommand{\paragraph}{
    \@ifstar
      \xxxParagraphStar
      \xxxParagraphNoStar
  }
  \newcommand{\xxxParagraphStar}[1]{\oldparagraph*{#1}\mbox{}}
  \newcommand{\xxxParagraphNoStar}[1]{\oldparagraph{#1}\mbox{}}
\fi
\ifx\subparagraph\undefined\else
  \let\oldsubparagraph\subparagraph
  \renewcommand{\subparagraph}{
    \@ifstar
      \xxxSubParagraphStar
      \xxxSubParagraphNoStar
  }
  \newcommand{\xxxSubParagraphStar}[1]{\oldsubparagraph*{#1}\mbox{}}
  \newcommand{\xxxSubParagraphNoStar}[1]{\oldsubparagraph{#1}\mbox{}}
\fi
\makeatother

\usepackage{color}
\usepackage{fancyvrb}
\newcommand{\VerbBar}{|}
\newcommand{\VERB}{\Verb[commandchars=\\\{\}]}
\DefineVerbatimEnvironment{Highlighting}{Verbatim}{commandchars=\\\{\}}
% Add ',fontsize=\small' for more characters per line
\usepackage{framed}
\definecolor{shadecolor}{RGB}{241,243,245}
\newenvironment{Shaded}{\begin{snugshade}}{\end{snugshade}}
\newcommand{\AlertTok}[1]{\textcolor[rgb]{0.68,0.00,0.00}{#1}}
\newcommand{\AnnotationTok}[1]{\textcolor[rgb]{0.37,0.37,0.37}{#1}}
\newcommand{\AttributeTok}[1]{\textcolor[rgb]{0.40,0.45,0.13}{#1}}
\newcommand{\BaseNTok}[1]{\textcolor[rgb]{0.68,0.00,0.00}{#1}}
\newcommand{\BuiltInTok}[1]{\textcolor[rgb]{0.00,0.23,0.31}{#1}}
\newcommand{\CharTok}[1]{\textcolor[rgb]{0.13,0.47,0.30}{#1}}
\newcommand{\CommentTok}[1]{\textcolor[rgb]{0.37,0.37,0.37}{#1}}
\newcommand{\CommentVarTok}[1]{\textcolor[rgb]{0.37,0.37,0.37}{\textit{#1}}}
\newcommand{\ConstantTok}[1]{\textcolor[rgb]{0.56,0.35,0.01}{#1}}
\newcommand{\ControlFlowTok}[1]{\textcolor[rgb]{0.00,0.23,0.31}{\textbf{#1}}}
\newcommand{\DataTypeTok}[1]{\textcolor[rgb]{0.68,0.00,0.00}{#1}}
\newcommand{\DecValTok}[1]{\textcolor[rgb]{0.68,0.00,0.00}{#1}}
\newcommand{\DocumentationTok}[1]{\textcolor[rgb]{0.37,0.37,0.37}{\textit{#1}}}
\newcommand{\ErrorTok}[1]{\textcolor[rgb]{0.68,0.00,0.00}{#1}}
\newcommand{\ExtensionTok}[1]{\textcolor[rgb]{0.00,0.23,0.31}{#1}}
\newcommand{\FloatTok}[1]{\textcolor[rgb]{0.68,0.00,0.00}{#1}}
\newcommand{\FunctionTok}[1]{\textcolor[rgb]{0.28,0.35,0.67}{#1}}
\newcommand{\ImportTok}[1]{\textcolor[rgb]{0.00,0.46,0.62}{#1}}
\newcommand{\InformationTok}[1]{\textcolor[rgb]{0.37,0.37,0.37}{#1}}
\newcommand{\KeywordTok}[1]{\textcolor[rgb]{0.00,0.23,0.31}{\textbf{#1}}}
\newcommand{\NormalTok}[1]{\textcolor[rgb]{0.00,0.23,0.31}{#1}}
\newcommand{\OperatorTok}[1]{\textcolor[rgb]{0.37,0.37,0.37}{#1}}
\newcommand{\OtherTok}[1]{\textcolor[rgb]{0.00,0.23,0.31}{#1}}
\newcommand{\PreprocessorTok}[1]{\textcolor[rgb]{0.68,0.00,0.00}{#1}}
\newcommand{\RegionMarkerTok}[1]{\textcolor[rgb]{0.00,0.23,0.31}{#1}}
\newcommand{\SpecialCharTok}[1]{\textcolor[rgb]{0.37,0.37,0.37}{#1}}
\newcommand{\SpecialStringTok}[1]{\textcolor[rgb]{0.13,0.47,0.30}{#1}}
\newcommand{\StringTok}[1]{\textcolor[rgb]{0.13,0.47,0.30}{#1}}
\newcommand{\VariableTok}[1]{\textcolor[rgb]{0.07,0.07,0.07}{#1}}
\newcommand{\VerbatimStringTok}[1]{\textcolor[rgb]{0.13,0.47,0.30}{#1}}
\newcommand{\WarningTok}[1]{\textcolor[rgb]{0.37,0.37,0.37}{\textit{#1}}}

\usepackage{longtable,booktabs,array}
\usepackage{calc} % for calculating minipage widths
% Correct order of tables after \paragraph or \subparagraph
\usepackage{etoolbox}
\makeatletter
\patchcmd\longtable{\par}{\if@noskipsec\mbox{}\fi\par}{}{}
\makeatother
% Allow footnotes in longtable head/foot
\IfFileExists{footnotehyper.sty}{\usepackage{footnotehyper}}{\usepackage{footnote}}
\makesavenoteenv{longtable}
\usepackage{graphicx}
\makeatletter
\newsavebox\pandoc@box
\newcommand*\pandocbounded[1]{% scales image to fit in text height/width
  \sbox\pandoc@box{#1}%
  \Gscale@div\@tempa{\textheight}{\dimexpr\ht\pandoc@box+\dp\pandoc@box\relax}%
  \Gscale@div\@tempb{\linewidth}{\wd\pandoc@box}%
  \ifdim\@tempb\p@<\@tempa\p@\let\@tempa\@tempb\fi% select the smaller of both
  \ifdim\@tempa\p@<\p@\scalebox{\@tempa}{\usebox\pandoc@box}%
  \else\usebox{\pandoc@box}%
  \fi%
}
% Set default figure placement to htbp
\def\fps@figure{htbp}
\makeatother



\ifLuaTeX
\usepackage[bidi=basic]{babel}
\else
\usepackage[bidi=default]{babel}
\fi
% get rid of language-specific shorthands (see #6817):
\let\LanguageShortHands\languageshorthands
\def\languageshorthands#1{}


\setlength{\emergencystretch}{3em} % prevent overfull lines

\providecommand{\tightlist}{%
  \setlength{\itemsep}{0pt}\setlength{\parskip}{0pt}}



 


\KOMAoption{captions}{tableheading}
\makeatletter
\@ifpackageloaded{caption}{}{\usepackage{caption}}
\AtBeginDocument{%
\ifdefined\contentsname
  \renewcommand*\contentsname{Obsah}
\else
  \newcommand\contentsname{Obsah}
\fi
\ifdefined\listfigurename
  \renewcommand*\listfigurename{Zoznam obrázkov}
\else
  \newcommand\listfigurename{Zoznam obrázkov}
\fi
\ifdefined\listtablename
  \renewcommand*\listtablename{Zoznam tabuliek}
\else
  \newcommand\listtablename{Zoznam tabuliek}
\fi
\ifdefined\figurename
  \renewcommand*\figurename{Obrázok}
\else
  \newcommand\figurename{Obrázok}
\fi
\ifdefined\tablename
  \renewcommand*\tablename{Tabuľka}
\else
  \newcommand\tablename{Tabuľka}
\fi
}
\@ifpackageloaded{float}{}{\usepackage{float}}
\floatstyle{ruled}
\@ifundefined{c@chapter}{\newfloat{codelisting}{h}{lop}}{\newfloat{codelisting}{h}{lop}[chapter]}
\floatname{codelisting}{Výpis}
\newcommand*\listoflistings{\listof{codelisting}{Zoznam výpisov}}
\makeatother
\makeatletter
\makeatother
\makeatletter
\@ifpackageloaded{caption}{}{\usepackage{caption}}
\@ifpackageloaded{subcaption}{}{\usepackage{subcaption}}
\makeatother
\usepackage{bookmark}
\IfFileExists{xurl.sty}{\usepackage{xurl}}{} % add URL line breaks if available
\urlstyle{same}
\hypersetup{
  pdftitle={INFORMATÍVNY MONITORING KVALITY OVZDUŠIA POMOCOU NÍZKO NÁKLADOVÝCH SENZOROV},
  pdfauthor={Petronela Ševčíková},
  pdflang={sk},
  colorlinks=true,
  linkcolor={blue},
  filecolor={Maroon},
  citecolor={Blue},
  urlcolor={Blue},
  pdfcreator={LaTeX via pandoc}}


\title{INFORMATÍVNY MONITORING KVALITY OVZDUŠIA POMOCOU NÍZKO
NÁKLADOVÝCH SENZOROV}
\usepackage{etoolbox}
\makeatletter
\providecommand{\subtitle}[1]{% add subtitle to \maketitle
  \apptocmd{\@title}{\par {\large #1 \par}}{}{}
}
\makeatother
\subtitle{LIFE-IP Aktivita D - časť: SENZORY}
\author{Petronela Ševčíková}
\date{}
\begin{document}
\maketitle

\renewcommand*\contentsname{Obsah}
{
\hypersetup{linkcolor=}
\setcounter{tocdepth}{3}
\tableofcontents
}

\pandocbounded{\includegraphics[keepaspectratio]{Populair+Life.jpg}}

\section{Slovenský Hydrometeorologický
Ústav}\label{slovenskuxfd-hydrometeorologickuxfd-uxfastav}

\subsection{Záverečná správa}\label{zuxe1vereux10dnuxe1-spruxe1va}

\subsubsection{December 2027}\label{december-2027}

Spolufinancované Európskou úniou. Vyjadrené názory a stanoviská sú
výlučne názormi autora/autorky/autorov a nemusia nevyhnutne odrážať
názory Európskej únie alebo CINEA. Európska únia ani orgán poskytujúci
grant za ne nenesú zodpovednosť.

Projekt je spolufinancovaný z prostriedkov štátneho rozpočtu SR
prostredníctvom Ministerstva životného prostredia SR.

\pandocbounded{\includegraphics[keepaspectratio]{Loga-2-riadky.png}}

\subsection{Úvod}\label{uxfavod}

\subsection{Zoznam skratiek a
značiek}\label{zoznam-skratiek-a-znaux10diek}

List of Abbreviations

SHMÚ: Slovenský Hydrometeorologický Ústav

EU: Európska Únia

SR: Slovenská Republika

CINEA: European Climate, Infrastructure and Environment Executive Agency

MKO: Manažér kvality ovzdušia

PM\textsubscript{2.5}: Znečisťujúca látka pevné častice a koloidy do
veľkosti 2.5 mikrometra

PM\textsubscript{10}: Znečisťujúca látka pevné častice a koloidy do
veľkosti 10 mikrometrov

\subsection{Výber lokalít pre monitorovanie
LCS}\label{vuxfdber-lokaluxedt-pre-monitorovanie-lcs}

Sumarizácia postupu a použitých kritérií pre výber monitorovanej
lokality

Ozrejmenie počtu lokalít na jednotlivý kraj

Opis umiestňovania zariadení v rámci jednotlivej lokality a jeho
zdôvodnenie (exp/poz)

\subsection{Metodika monitorovania a zabezpečenia kvality meraných dát v
teréne}\label{metodika-monitorovania-a-zabezpeux10denia-kvality-meranuxfdch-duxe1t-v-teruxe9ne}

Opis vybranej metódy referencovania meraní (40d súmernaia, počet
zariadení, počet lokalít, počet súmeraní)

Kritériá pre výber a rozdelovanie zariadení na lokality, ktotériá pre
výber a pridelenie rREF staníc pre jednotlivé lokality

\subsection{Metodika korekcií}\label{metodika-korekciuxed}

\subsection{Metodika validácie}\label{metodika-validuxe1cie}

\subsection{Metodika vyhodnotenia
dát}\label{metodika-vyhodnotenia-duxe1t}

Opis metódy validácia meraní od zberu dát, vyhlásenia za platné dáta,
vyhlásenia platného priemeru za časové obdobie a podobne.

Vyhodnotenie presnosti a správnosti dát a opis metodiky nasadenia
korekčného faktora

Opis metódy priebežnej kontroly kvality meraní na experimentálnych
lokalitách (toto ešte musíme vymyslieť)

Dostupnosť RAW a korigovaných dát (ideálne by mohla byť OPEN)

\subsection{Použité prístroje}\label{pouux17eituxe9-pruxedstroje}

Opis prístrojov v rámci referenčnej stanice, stanovenie ich presnosti
správnosti (ak je dostupné)

Opis experimentálnych zariadení 1 (od dodávateľa)

Opis experimentálnych zariadení 2 (vlastné) -- tu sa dohodnúť, koľko
detailov chceme zdieľať.

Opis všetkých ďalších pomocných zariadení, ak boli použité

\subsection{Referenčné stanice (Plášťovce,
Ružomberok)}\label{referenux10dnuxe9-stanice-pluxe1ux161ux165ovce-ruux17eomberok}

Charakteristika stanice, znečistenia, typických priemerných
ročných/sezónnych koncentrácií.

Dokumentácia priebehu súmeraní na začiatku a na konci q monitorovania
(ak prebehlo)

Vyhodnotenie súmeraní Ref vs.~Senzor, stanovenie korekčných faktorov,

\subsection{Výsledky monitorovania pre všetky monitorované
lokality}\label{vuxfdsledky-monitorovania-pre-vux161etky-monitorovanuxe9-lokality}

možno vo forme rastrových máp s interpolovanými priemernými
ročnými/kvartálnymi koncentráciami so zobrazením hraníc krajov

porovnanie krajov navzájom (priemery na lokalitách)

nájsť vhodné grafické zobrazenie

Porovnanie výsledkov z príslušnej REF stanice a klastra senzorov na
zodpovedajúcich EXP lokalitcáh

Porovnanie výsledkov meraní senzorov s výsledkami z najbližšej podobnej
stanice NMSKO //predebatujeme//

\subsection{Analýza detekcie driftu meraní LCS počas
monitorovania}\label{analuxfdza-detekcie-driftu-meranuxed-lcs-poux10das-monitorovania}

Vyhodnotenie dvoch referenčných súmeraní (na začiatku a na konci
monitorovania)

\subsection{Porovnanie Modelov Kvality ovzdušia a meraných dát na
jednotlivých
lokalitách}\label{porovnanie-modelov-kvality-ovzduux161ia-a-meranuxfdch-duxe1t-na-jednotlivuxfdch-lokalituxe1ch}

Ak to dáva zmysel: Porovnanie rastrov z modelov a rastrov z
interpolovaných meraní Resp. porovnanie koncentrácií zobrazených v
modeli (napr. Rio alebo aj inom) bez započítania koncentrácií s LCS
monitorovania vrátane započítania týchto dát (toto bude asi lepšie) //tu
budeme musieť osloviť modelárov a spýtať sa, či by chceli na tom
spolupracovať oproti dohode//

\subsection{Vyhodnotenie využiteľnosti LCS pre monitorovanie kvality
ovzdušia}\label{vyhodnotenie-vyuux17eiteux13enosti-lcs-pre-monitorovanie-kvality-ovzduux161ia}

Kritické zhodnotenie dosiahnutých výsledkov, zhrnutie výsledkov oproti
cieľom projektu, a perspektívy do budúcnosti

\subsection{Záverečné zhrnutie}\label{zuxe1vereux10dnuxe9-zhrnutie}

Krátky záver sumarizujúci práce na projekte

\subsection{Literatúra}\label{literatuxfara}

\subsubsection{Appendix I \# Čiastková záverečná správa pre každú
lokalitu}\label{appendix-i-ux10diastkovuxe1-zuxe1vereux10dnuxe1-spruxe1va-pre-kaux17eduxfa-lokalitu}

interaktívna mapa so všetkými lokalitami, a ty si klikneš na guličku a
vyroluje sa ti záverečná správa

\subsubsection{Appendix II \# Záznamy o zverejňovaní dát (SAZP/weby
obcí/SHMU(?))}\label{appendix-ii-zuxe1znamy-o-zverejux148ovanuxed-duxe1t-sazpweby-obcuxedshmu}

\subsubsection{\texorpdfstring{Appendix III \# Dostupnosť dát Appendix
IV\\
}{Appendix III \# Dostupnosť dát Appendix IV }}\label{appendix-iii-dostupnosux165-duxe1t-appendix-iv}

\subsubsection{Záznam z outreach aktivít pre verejnosť -- ak
relevantné}\label{zuxe1znam-z-outreach-aktivuxedt-pre-verejnosux165-ak-relevantnuxe9}

\subsubsection{Appendix V \# Zoznam publikácií okrem Záverečnej
Správy}\label{appendix-v-zoznam-publikuxe1ciuxed-okrem-zuxe1vereux10dnej-spruxe1vy}

\subsection{Running Code}\label{running-code}

When you click the \textbf{Render} button a document will be generated
that includes both content and the output of embedded code. You can
embed code like this:

\begin{Shaded}
\begin{Highlighting}[]
\DecValTok{1} \SpecialCharTok{+} \DecValTok{1}
\end{Highlighting}
\end{Shaded}

\begin{verbatim}
[1] 2
\end{verbatim}

You can add options to executable code like this

\begin{verbatim}
[1] 4
\end{verbatim}

The \texttt{echo:\ false} option disables the printing of code (only
output is displayed).

:::: :::::




\end{document}
